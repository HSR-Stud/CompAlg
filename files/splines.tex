\section{Spline-Interpolation}

\begin{minipage}[c]{14.5cm}
Die Idee der Spline-Interpolation ist, die Daten nicht mit einem Polynom hohen Grades zu interpolieren, sondern mit mehreren Polynomen tiefen Grades. Daduch kann die Tendenz  zum Schwingen von Polynomen hohen Grades umgangen werden. Bei den "Bruchstellen", den Übergängen von einem Patch zum nächsten, müssen die Ableitungen der benachbarten Patches bis zu einem vorgegeben Ableitung übereinstimmten.
\end{minipage}
\hfill
\begin{minipage}[c]{4cm}
\includegraphics[width=\textwidth]{bilder/kubikSpline}
\end{minipage}

\textbf{Eigenschaften}
\begin{liste}
  \item[\textbf{+}] Kein Runge-Phänomen (Schwingungen am Rand)
  \item[\textbf{+}] Polynome haben tiefen Grad
  \item[\textbf{+}] Aufwand zur Berechnung geringer als Newton $O(n)$ statt $O(n^2)$  
  \item[$\mathbf{-}$] "`Nicht eingebettet"' (neue Messungen bedeutet, neue Polynomberechnung!)
  \item[$\mathbf{-}$] Polynome müssen zur Auswertung zusammengesetzt werden (Post-Processing Aufwand gross)
\end{liste}


\subsection{Ein-Dimensionale Splines}
\subsubsection{Prinzip}
Pro \em Patch \em (von total $n$) wird ein Polynom vom Grad $d$ (meist kubisch, $d=3$) berechnet. An den Übergängen 
bei $x_i$ können Bedingungen $C^k$ ($k=1,\ldots d-1$) definiert werden.

\paragraph{Freiheitsgrade}
$n$ Patches mit je 1 Polynom mit je $(d+1)$ Koeffizienten, gibt total $n(d+1)$ Freiheitsgrade.

\paragraph{Bedingungen}
$n$ Patches mit Anfang und Ende $=2n$;
$d-1$ Ableitungen pro innerem Punkt $(n-1)(d-1)$ ergibt total ($(n(d+1)-(d-1))$) Bedingungen.

\paragraph{Zusatzbedingungen}

Vergleicht man Freiheitsgrade und Bedinungen, sind $d-1$ Bedingungen zu wählen. Beim \em natural 
spline \em werden dazu die zweiten Ableitungen auf 0 gesetzt; beim \em clamped splines \em werden
die 1. Ableitungen vorgegeben.


\subsubsection{Allgemeines Vorgehen (kubische Splines)}
Die endgültige Interpolation hat diese Form:
$$\boxed{S_i(x) = a_i + b_i(x-x_i) + c_i(x-x_i)^2 + d_i(x-x_i)^3}.$$
Gebraucht wird auch $h_i = x_{i+1} - x_i = \Delta x_i$ 
\begin{enumerate}
  \item $a_i = y_i \qquad (i=0,\ldots n-1)$
  \item Für kubische Splines hat es $d-1=3-1=2$ Zusatzbedingungen.
    \begin{itemize}
      \item Natural Splining:
      $y_0''=0=y_n''$
        \begin{enumerate}
          \item $c_0 = c_n = 0$
          \item Gleichungssystem nach $c_i$ auflösen:\\
            \includegraphics[width=13cm]{./bilder/1d_spline_natural_gleichungssystem}
          \item $d_{n-1} = -\frac{c_{n-1}}{3h_{n-1}}$
        \end{enumerate}
        
       \item Clamped Splining:
         \begin{enumerate}
           \item $b_0 = y_0'; b_n=y_n'$
           \item Gleichungssystem nach $c_i$ auflösen:\\
             \includegraphics[width=12cm]{./bilder/1d_spline_gleichungssystem}
           \item $d_{n-1} = \frac{y_n' - b_{n-1} - 2c_{n-1}h_{n-1}}{3h_{n-1}^2}$ (am Schluss zu berechnen)
         \end{enumerate}
    \end{itemize}
  \item $b_{i-1} = \frac{a_i - a_{i-1}}{h_{i-1}} - \frac{2 c_{i-1} + c_i}{3} h_{i-1} \qquad (i=1, \ldots n-1)$
  \item $b_{n-1} = \frac{y_n - a_{n-1}}{h_{n-1}} - c_{n-1} h_{n-1} - d_{n-1}h_{n-1}^2=\frac{y_n-y_{n-1}}{h_{n-1}}-\frac 23 c_{n-1}h_{n-1}$
  \item $d_{i-1} = \frac{c_i - c_{i-1}}{3 h_{i-1}} \qquad (i=1, \ldots n-1)$
  \item Bei clamped splining: $d_{n-1} = \frac{y_n' - b_{n-1} - 2c_{n-1}h_{n-1}}{3h_{n-1}^2}$
\end{enumerate}

\subsubsection{Fehlerabschätzung}
Für kubische Splines mit $C^2$ gilt folgende Fehlerabschätzung ($H = \max h_i \; (i=0, \ldots n-1)$):\\
$| y(x) - S(x) | \leq \max |y^{(4)}(x)| \frac{5}{384} H^4 \qquad
 | y'(x) - S'(x) | \leq \max |y^{(4)}(x)| \frac{1}{24} H^3 \qquad 
 | y''(x) - S''(x) | \leq \max |y^{(4)}(x)| \frac{3}{8} H^2$
 
 

\subsection{Bernstein-Bézier Splines (B-B Splines)}

\begin{minipage}[c]{15.0cm}	
	\subsubsection{Bernstein-Polynome}
	  	Bereich $[0,1]$:\\
	  	
	    $B_{i,n}(t) = \binom{n}{i}(1-t)^{n-i} t^i \qquad t \in [0,1]\; i=0,1,\ldots, n$\\
	    
	    Bereich $[a,b]$:\\
	    
	    $B_{i,n}(u,a,b) =\frac{1}{(b-a)^n}\binom{n}{i}(b-u)^{n-i} (u-a)^i \qquad u \in [a,b]\; i=0,1,\ldots, n$
	\paragraph{Eigenschaften}
		Bernstein Polynome$\ldots$
	    \begin{itemize}
	      \item ergeben eine lineare Basis für Polynome der Ordnung $n$ (man kann mit ihnen jedes Polynom der Ordnung $n$ zusammenbauen)
	      \item haben genau eine Maximalstelle bei $t=\frac in$
	      \item haben eine Nullstelle bei $0$ (Ordnung $i$) und bei $1$ (Ordnung $n-1$)
	      \item sind symmetrisch: $B_{i,n}(t) = B_{n-i,n}(1-t)$
	      \item sind zwischen $t \in [0,1]$ begrenzt.
	      \item ergeben in der Summe: $\sum \limits_{i=0}^n B_{i,n}(t)=1$
	    \end{itemize}
	    
	    $\frac{d}{dt} B_{i,n}(t) = n(B_{i-1,n-1}(t) - B_{i,n-1}(t)) = -n \Delta B_{i-1,n-1}(t)$\\
	    $\frac{d^2}{dt^2} B_{i,n}(t) = n(n-1)(B_{i-2,n-2}(t) -2 B_{i-1,n-2}(t) + B_{i,n-2}(t)) = -n(n-1) \Delta^2 B_{i-2,n-2}(t)$\\
	    $\frac{d^k}{dt^k} B_{i,n}(t) = (-1)^k n(n-1)\ldots(n-k+1) \Delta^k B_{i-k,n-k}(t)$
\end{minipage}
\hfill    
\begin{minipage}[c]{4cm}
  	\includegraphics[width=\textwidth]{bilder/bernsteinBezier}
\end{minipage}

\vspace{0.2cm}

\begin{minipage}[c]{5.5cm}	
\textbf{Pascalsches Dreieck}\\

	\scalebox{0.8}{
	\renewcommand{\arraystretch}{1}
	\begin{tabular}{rccccccccc}
		$n=0$:&    &    &    &    &  1\\\noalign{\smallskip\smallskip}
		$n=1$:&    &    &    &  1 &    &  1\\\noalign{\smallskip\smallskip}
		$n=2$:&    &    &  1 &    &  2 &    &  1\\\noalign{\smallskip\smallskip}
		$n=3$:&    &  1 &    &  3 &    &  3 &    &  1\\\noalign{\smallskip\smallskip}
		$n=4$:&  1 &    &  4 &    &  6 &    &  4 &    &  1\\\noalign{\smallskip\smallskip}
	\end{tabular}}
\end{minipage}
\hfill    
\begin{minipage}[c]{13.25cm}   
\textbf{Bernstein Polynome} \quad $0\leq t\leq 1$\\

	\scalebox{0.8}{
	\begin{tabular}{lllll}
		$B_{0,0}(t)=1$&&&&\\
		$B_{0,1}(t)=1-t$&$B_{1,1}(t)=t$&&&\\
		$B_{0,2}(t)=(1-t)^2$&$B_{1,2}(t)=2t(1-t)$&$B_{2,2}(t)=t^2$&&\\
		$B_{0,3}(t)=(1-t)^3$&$B_{1,3}(t)=3t(1-t)^2$&$B_{2,3}(t)=3t^2(1-t)$&$B_{3,3}(t)=t^3$&\\
		$B_{0,4}(t)=(1-t)^4$&$B_{1,4}(t)=4t(1-t)^3$&$B_{2,4}(t)=6t^2(1-t)^2$&$B_{3,4}(t)=4t^3(1-t)$&$B_{4,4}(t)=t^4$\\
	\end{tabular}}
\renewcommand{\arraystretch}{1.5} 
\end{minipage}  


\subsubsection{Simple Bézier Kurven}
Eine Bézier-Kurve wird über Kontrollpunkte ($\vec{P}_0, \vec{P}_1, \ldots \vec{P}_n\,(n \geq 2)$ 
in $R^d$) sowie die Bernstein Polynome definiert:
$$\vec{r}(t) = \sum \limits_{i=0}^{n} \vec{P}_i B_{i,n}(t) \quad t \in [0,1]$$
\paragraph{Eigenschaften}
\begin{itemize}
  \item Die Bézier-Kurven liegen immer innerhalb der konvexen Hülle der Kontrollpunkte.
  \item $\vec{r}(0) = \vec{P}_0$, 
        $\qquad \vec{r}(1) = \vec{P}_n$
  \item $\vec{r} '(0) = n(\vec{P}_1-\vec{P}_0)$, 
        $\qquad \vec{r}'(1) = n(\vec{P}_n-\vec{P}_{n-1})$
  \item $\vec{r} ''(0) = n(n-1)(\vec{P}_2 - 2\vec{P}_1 + \vec{P}_0)$,
        $\qquad \vec{r} ''(1) = n(n-1)(\vec{P}_n - 2\vec{P}_{n-1} + \vec{P}_{n-2})$
  \item Wenn $C^k$ Übergang zu einem Punkt gefordert ist, sind $k$ Gleichungen bzw. $k$ 
        Kontrollpunkte pro Punkt nötig
\end{itemize}


\begin{minipage}{13.5cm}
  \subsubsection{Zusammengesetzte (Composite) Bézier Kurven}
  Die zusammengesetzten (simplen) Bézier-Kurven sollen die bekannte Bedingungen ($C^k$) erfüllen.
  Dies sind die Gleichungen um den Punkt $Q_0$ mit Kontrollpunkten $Q_{1,\ldots,m-1}$ nach $Q_0$ sowie
  den Kontrollpunkten $P_{1,\ldots,n-1}$ vor $Q_0$. 
  $$C^0: \quad \boxed{\vec{P_n} = \vec{Q_0}}$$
  $$C^1: \quad \vec{r}_P'(1) = \boxed{n(\vec{P}_n-\vec{P}_{n-1}) = m(\vec{Q}_1-\vec{Q_0})} = \vec{r}_Q'(0)$$
  $$C^2: \quad \vec{r}_P''(1) = \boxed{n(n-1)(\vec{P}_n - 2\vec{P}_{n-1} + \vec{P}_{n-2}) = 
    m(m-1)(\vec{Q}_2 - 2\vec{Q_1} + \vec{Q_0})} = \vec{r}_Q''(0)$$
  $m$ ist dabei der Grad von $\vec{r}_Q$, $n$ derjenige von $\vec{r}_P$.
\end{minipage}
\begin{minipage}{5.5cm}
  \includegraphics[width=5.5cm]{./bilder/composite_bezier.png}
  Hier sind $m=n=3$
\end{minipage}

\subsubsection{Vergleich Bézier- und Newton-Interpolation mit gleichverteilten Argumenten auf X-Achse}
  Aus Sicht Bézier:
  \begin{liste}
    \item[+] Kleiner Fehler, keine Oszillation, kein Runge (bei kleinem Grad, z.B. 3)
    \item[+] $O(n)$ (linearer Rechenaufwand)
    \item[-] Nicht eingebettet (neue Daten brauchen Neuberechnung der Interpolation)
    \item[-] Post-Processing (Grafik, Integrations, Ableitungen) sind komplexer, da jedes Kurvenstück
      einzeln betrachtet werden muss.
  \end{liste}
