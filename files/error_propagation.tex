\section{Fehlerfortpflanzung}

\subsection{Differential}
  Das Differential bezeichnet den linearen Anteil des Zuwachses einer Variablen einer Funktion.
  $$\Delta y \approx df = f'(x_o) dx = f'(x_0) h = f'(x_0) \Delta x$$
  
  Jede $n$-fach ableitbare Funktion kann mit einer \textbf{Taylor}-Reihe approximiert werden:
  $$f(x_0+h) \approx f(x_0) + h f'(x_0) + \frac{h^2}{2} f''(x_0) + \frac{h^3}{3!} f'''(x_0) + 
  \ldots + \frac{h^n}{n!} f^{(n)} + R_n(x_0,h)$$
  Wobei $R_n(x_0,h)$ das Restglied bezeichnet und mit $n \rightarrow \infty$ gegen $0$ mit 
  Geschwindigkeit $o(h^n)$ ("`schnell"') konvergiert. Es gibt bspw. die 
  \begin{align}
    \text{Lagrange Form: } R_n(x_0,h) &= \frac{h^{n+1}}{(n+1)!} f^{(n+1)}(\underbrace{x_0 + \vartheta h}_{\xi})
    \qquad (\xi \in (x_0, x_0+h), \vartheta \in (0,1)) \qquad \text{oder die }\\
    \text{Cauchy Form: } R_n(x_0,h) &= \frac{h^n(1-\vartheta)^n}{(n+1)!} f^{(n+1)}(\underbrace{x_0 + \vartheta h}_{\xi})
    \qquad (\xi \in (x_0, x_0+h), \vartheta \in (0,1))
  \end{align}
  
  
   
\subsection{Multivariate Differentiale, Taylor}
  Das Differential sieht im Multidimensionalen so aus:
  $$\Delta f \approx df(\vec{x}) = h_1 \frac{\partial f(\vec{x})}{\partial x_1} + \ldots + 
  h_n \frac{\partial f(\vec{x})}{\partial x_n} \qquad \vec{h} = [h_1, \ldots, h_d]^T, \; \vec{x} = [x_1, \ldots x_d]^T$$

  Und Taylor für 2D:
  $$f(\vec{x}_0 + \vec{h}) \approx f(\vec{x}_0) + \vec{h} \nabla f(\vec{x}_0) +
  \underbrace{\bigg( \frac{h_1^2}{2!} \frac{\partial^2 f}{\partial x_1^2} + \frac{h_2^2}{2!} \frac{\partial^2 f}{\partial^2 x_2^2}
  + \frac{h_1 h_2}{1!1!} \frac{\partial^2 f}{\partial x_1 \partial x_2}\bigg)}_{\text{für 2. Ordnung}} + R_2(\vec{x}_0, \vec{h})$$

\subsection{Jacobi-Matrix}
  \begin{minipage}{12cm}
    Die Jacobi-Matrix bildet alle ersten Ableitungen einer Funktion ab.
    Wenn die Jacobi-Matrix quadratisch ist ($m=n$), dann kann dessen Determinante $\det(J)$ als 
    transformiertes Volumen interpretiert werden:
    $$\underbrace{dy_1 dy_2 \ldots dy_n}_{\text{"`transformiertes Volumenelement"'}} = 
    \det(J_f(\vec{x})) \underbrace{dx_1 dx_2 \ldots dx_n}_{\text{"`Volumenelement"'}}$$
    Die Jacobi-Determinante kann daher auch als ein Mass für Fehlerfortpflanzung sein.
    Die Berechnung erfolgt direkt über die Determinanten-Regel, über 
    $$\frac{D(f_1, f_2, \ldots,f_n)}{D(x_1, x_2, \ldots, x_n)} = \det(J_f(\vec{x})) \quad \text{oder} \quad
    \det(J_f) = \sqrt{\det \Big( \underbrace{J_f^T(\vec{x}) J_f(\vec{x})}_{\text{Diagonalmatrix}} \Big)}.$$
    
    Ausserdem gilt für die Transformation $T$ und deren Inverse $T^{-1}$: $\det(J_T) = \frac{1}{\det{J_{T^{-1}}}}$
  \end{minipage}
  \hspace{5mm}
  \begin{minipage}{6cm}
    $$J_f(\vec{x}) :=  \begin{bmatrix}
    \frac{\partial f_1}{\partial x_1}(\vec{x}) & \frac{\partial f_1}{\partial x_2}(\vec{x}) & \ldots & \frac{\partial f_1}{\partial x_n}(\vec{x}) \\
    \vdots & \vdots & \ddots & \vdots \\
    \frac{\partial f_m}{\partial x_1}(\vec{x}) & \frac{\partial f_m}{\partial x_2}(\vec{x}) & \ldots & \frac{\partial f_m}{\partial x_n} (\vec{x})
    \end{bmatrix}$$
  \end{minipage}
  